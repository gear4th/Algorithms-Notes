\documentclass[a4paper,12pt]{report}

\setlength\parindent{0pt}
\usepackage{subfiles}
\usepackage{amsmath}
\usepackage{hyperref}
\usepackage{mathtools}
\usepackage{listings}
\usepackage{xcolor}
\lstset { %
    language=C++,
    backgroundcolor=\color{black!5}, % set backgroundcolor
    basicstyle=\footnotesize,% basic font setting
    tabsize=3,
    breaklines=true
}
\DeclarePairedDelimiter\ceil{\lceil}{\rceil}
\DeclarePairedDelimiter\floor{\lfloor}{\rfloor}
\newcommand*{\Perm}[2]{{}^{#1}\!P_{#2}}%
\newcommand*{\Comb}[2]{{}^{#1}C_{#2}}%
\DeclarePairedDelimiter\abs{\lvert}{\rvert}%
\DeclarePairedDelimiter\norm{\lVert}{\rVert}%
\makeatletter
\let\oldabs\abs
\def\abs{\@ifstar{\oldabs}{\oldabs*}}
%
\let\oldnorm\norm
\def\norm{\@ifstar{\oldnorm}{\oldnorm*}}
\makeatother

\begin{document}

\font\TitleF=cmr12 at 40pt
\title{\TitleF Algorithms Notes}
\author{Bliss Of Comprehension}
\date{\today}
\maketitle

\tableofcontents

\chapter*{preface}

This book is largely concerned with algorithms helpful for solving OJ problems.\\
An algorithmic way of approaching codeforces problems.
\begin{itemize}
	\item Read and understand the problem statement.
	\item Read input format.
	\item Verify sample tests with brute force to check your understanding of problem statement.
	\item ReRead Bold words in the problem statement.
	\item Check the time limit.(may be you can use sqrt decomp etc)
	\item Come up with algorithm.
	\item Verify the sample tests with your algorithm.
\end{itemize}

\subfile{stack/stack}
\subfile{XOR/XOR}
\subfile{counting/counting}
\subfile{graphs/graphs}
\subfile{codetricks/codetricks}
\subfile{arrays/arrays}
\subfile{numbers/numbers}
\subfile{dp/dp}
\subfile{adhoc/adhoc}

\end{document}
