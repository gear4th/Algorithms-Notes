\documentclass[a4paper,12pt]{report}

\setlength\parindent{0pt}
\usepackage{subfiles}
\usepackage{amsmath}
\usepackage{hyperref}
\usepackage{mathtools}
\usepackage{listings}
\usepackage{xcolor}
\usepackage{graphicx}
\lstset { %
    language=C++,
    backgroundcolor=\color{black!5}, % set backgroundcolor
    basicstyle=\footnotesize,% basic font setting
    tabsize=3,
    breaklines=true
}
\DeclarePairedDelimiter\ceil{\lceil}{\rceil}
\DeclarePairedDelimiter\floor{\lfloor}{\rfloor}
\newcommand*{\Perm}[2]{{}^{#1}\!P_{#2}}%
\newcommand*{\Comb}[2]{{}^{#1}C_{#2}}%
\DeclarePairedDelimiter\abs{\lvert}{\rvert}%
\DeclarePairedDelimiter\norm{\lVert}{\rVert}%
\makeatletter
\let\oldabs\abs
\def\abs{\@ifstar{\oldabs}{\oldabs*}}
%
\let\oldnorm\norm
\def\norm{\@ifstar{\oldnorm}{\oldnorm*}}
\makeatother

\begin{document}

\font\TitleF=cmr12 at 40pt
\title{\TitleF Algorithms Notes}
\author{Bliss Of Comprehension}
\date{\today}
\maketitle

\tableofcontents

\chapter*{preface}

This book is largely concerned with algorithms helpful for solving OJ problems.\\
An algorithmic way of approaching codeforces problems.
\begin{itemize}
	\item Read and understand the problem statement.
	\item Read input format.
	\item Verify sample tests with brute force to check your understanding of problem statement.
	\item ReRead Bold words in the problem statement.
	\item Check the time limit.(may be you can use sqrt decomp etc)
	\item Come up with algorithm.
	\item Verify the sample tests with your algorithm.
	\item Think of an implementation. Rigorously try to \textbf{IMPROVE} your implementation, instead of coding on the first thought. 
\end{itemize}

Methods to check why your solution is giving wrong answer.
\begin{itemize}
	\item Go to input section, find the lowest and highest value a parameter takes and test your solution with that, for all parameters.
	\item May be it's not your cpp code that's wrong, but your algorithm itself. (You fucked up and wasted a lot of time by now)
\end{itemize}

Tricks to approach a problem
\begin{itemize}
	\item If the input only contains single number, Then the answer is probable in \textbf{OEIS}.
\end{itemize}

\subfile{stack/stack}
\subfile{XOR/XOR}
\subfile{counting/counting}
\subfile{graphs/graphs}
\subfile{codetricks/codetricks}
\subfile{arrays/arrays}
\subfile{numbers/numbers}
\subfile{dp/dp}
\subfile{adhoc/adhoc}
\subfile{strings/strings}

\end{document}
