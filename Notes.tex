\documentclass[a4paper,12pt]{report}

\setlength\parindent{0pt}
\usepackage{subfiles}
\usepackage{amsmath}
\usepackage{hyperref}
\usepackage{mathtools}
\usepackage{listings}
\usepackage{xcolor}
\lstset { %
    language=C++,
    backgroundcolor=\color{black!5}, % set backgroundcolor
    basicstyle=\footnotesize,% basic font setting
    tabsize=3,
    breaklines=true
}
\DeclarePairedDelimiter\ceil{\lceil}{\rceil}
\DeclarePairedDelimiter\floor{\lfloor}{\rfloor}
\newcommand*{\Perm}[2]{{}^{#1}\!P_{#2}}%
\newcommand*{\Comb}[2]{{}^{#1}C_{#2}}%
\DeclarePairedDelimiter\abs{\lvert}{\rvert}%
\DeclarePairedDelimiter\norm{\lVert}{\rVert}%
\makeatletter
\let\oldabs\abs
\def\abs{\@ifstar{\oldabs}{\oldabs*}}
%
\let\oldnorm\norm
\def\norm{\@ifstar{\oldnorm}{\oldnorm*}}
\makeatother

\begin{document}

\font\TitleF=cmr12 at 40pt
\title{\TitleF Algorithms Notes}
\author{Bliss Of Comprehension}
\date{\today}
\maketitle

\tableofcontents

\chapter*{preface}

This book is largely concerned with algorithms helpful for solving OJ problems.\\

Donot forget to use both \textbf{"LATERAL THINKING"} and "Vertical thinking" while solving problems.\\
Proof: All the editorials you have seen consists of a combination of max  2 to 3 previously known concepts or tricks. This implies that there is a path from start(problem statement) to goal(answer) which consists of atmost 3 edges, so simply think laterally untill you find a vertex or edge in that shortest path. (whose size is guaranteed to be $\leq 3$).\\ 

In some problems vertical thinking may produce lesser times than Lateral thinking, But Lateral is consistent in producing good times in all problems.\\

Consider two approaches of thinking.
\begin{itemize}
	\item Think of an idea and spend time on that. If answer not found 
	repeat.(normal thinking/ dfs type thinking)
	\item List down all the ideas initially(each from each step of 1st type of thinking) (ex 10 unrelated ideas). And sort the ideas by probability of reaching answer. Now spend time on each idea in sorted order.(Popularly known as "brainstorming").
\end{itemize}
Clearly 2nd way is better than 1st. And it is guarenteed to reach answer in less amount of time.\\

Let the algorithm space be a set of points(exact algorithms) and there is an edge between two points $\iff$ we can reach algo1 to algo2 with just a minamilistic tweak. 

\begin{itemize}
	\item All points(exact algorithms) related to a concept are really close together in the space
	\item An ideaset is set of all points(in which this particular idea is used).
	\item The exact solution(point) to a problem may really far away from our starting (point or ideaset).
	\item So starting with multiple points (or ideasets) may work faster in our search.
	\item we can also skip through multiple edges(instead of single edge each time) to reach our solution faster.
\end{itemize} 

Your brain is just a bunch of neurons. Combined states of billions of neurons represent a snapshot of your thoughts(your state). Similary thought about each idea can be roughly thought as a set of action potentials of all the neurons.

When you see a question some of the neurons fire based upon the  words and pictures in the question and your previous memory.(read about long term potentiation and hebbs law). (Initially neurons related to visual signal of words and pictures fire, but due to LLP(memory) some more neurons will fire) this gives rise to initial idea.

Initial idea differs by individuals previous experiences(what might be a creative and non obvious idea to you, might be someone's initial idea). And the mind goes on in a line of thought firing different neurons, but there won't be a drastic state change in the brain.

This is where creativity comes in to picture. Implement any thinker toy from thinker toys book, the state of the brain changes drastically. For example, associating a random word to our problem (neurons related to the "random word" will fire, so in conjuction with the problem neurons initially fired they combine to form a different state of the brain i.e thought of new idea).

Explicitly speaking, since you read thinker toys from the book, whenever you encounter a problem, neurons related to thinker toys book trigger, which inturn triggers a specific thinker toy essentials neurons(who might send impulses to eyes to look for a word and head to turn to the screen or book where words are present), and associates with problem neurons.\\

An algorithmic way of approaching codeforces problems.
\begin{itemize}
	\item Read and understand the problem statement.
	\item Read input format.
	\item Verify sample tests with brute force to check your understanding of problem statement.
	\item ReRead Bold words in the problem statement.
	\item Check the time limit.(may be you can use sqrt decomp etc)
	\item Come up with algorithm.
	\item Verify the sample tests with your algorithm.
	\item check accepted/wrong answer ratio. If it's too low then the obvious algorithms are wrong. Take time and recheck your algorithm.
	\item Think of an implementation. Rigorously try to \textbf{IMPROVE} your implementation, instead of coding on the first thought. 
\end{itemize}

It is really important to improve your speed. In a normal scenario, You have done A-D(0-1900 rated) questions.
\begin{itemize}
	\item  It can put you in rank (140- 960). For a rating of 1950 it is 
	(+90 or -45).
	\item And Simply decreasing each question time by 5min increases rating a lot, and leaves +20min for the 5th question.
\end{itemize}

Methods to check why your solution is giving wrong answer.
\begin{itemize}
	\item Go to input section, find the lowest and highest value a parameter takes and test your solution with that, for all parameters.
	\item May be it's not your cpp code that's wrong, but your algorithm itself. (You fucked up and wasted a lot of time by now)
\end{itemize}

Tricks to approach a problem
\begin{itemize}
	\item If the input only contains single number, Then the answer is probable in \textbf{OEIS}.
\end{itemize}

\subfile{stack/stack}
\subfile{XOR/XOR}
\subfile{counting/counting}
\subfile{graphs/graphs}
\subfile{codetricks/codetricks}
\subfile{arrays/arrays}
\subfile{numbers/numbers}
\subfile{dp/dp}
\subfile{adhoc/adhoc}
\subfile{strings/strings}

\end{document}
