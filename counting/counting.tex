\documentclass[../Notes.tex]{subfiles}

\begin{document}
\chapter{Counting}

\begin{itemize}
	\item Given a binary string, find number of 3-tuple (indices $i,j,k$ $\ni$ $i<j<k$) and exactly one of $s[i],s[j],s[k]$ is 1. 
	\item There are r boxes in a row and n identical balls. Find number of different ways we can distribute balls among boxes.\\
	$\Comb{n+r-1}{r-1}$
	\item In problems, to convert A to B using minimum operations. Looking at the final operation may help.
	\item When trying to solve non-linear recurrences using matrix expo.
	$f_i = f_{i - 1} + 2i^2 + 5$. The left column vectors should be
	$\begin{pmatrix} f_i\\ i^2\\ i\\ 1\end{pmatrix}$ and $\begin{pmatrix} f_{i-1}\\ (i-1)^2\\ i-1\\ 1\end{pmatrix}$ otherwise the matrices wouldn't be same and we cannot combine the equations.
	\item let $x_1+x_2+x_3+ \ldots +x_r = n$ and $x_i\leq \sqrt{n}$ then 
	$x_1^2+x_2^2+x_3^2+ \ldots +x_r^2 \leq n\sqrt{n}$.
	\item \textbf{Lucas Theorem} : $${\binom {m}{n}}\equiv \prod _{i=0}^{k}{\binom {m_{i}}{n_{i}}}{\pmod {p}}$$
	where $m=m_{k}p^{k}+m_{k-1}p^{k-1}+\cdots +m_{1}p+m_{0},$ and\\ 
	$n=n_{k}p^{k}+n_{k-1}p^{k-1}+\cdots +n_{1}p+n_{0}$ are base p representations of m and n respectively. Note that 
	$n_k<i_k\Longrightarrow\binom{n_k}{i_k}=0\Longrightarrow\binom{n}{i}\equiv 0 \pmod{p}$. This is equivalent to saying that there is no $x^i$ term in the expansion.
	\item The only best way to calculate ncr other than lucas is to have prefix facts and prefix inverse facts, then answer ncr in O(1).
	\item When n is really big and r is small, We can answer the query in O(r), without any preprocessing.
\end{itemize}
\section{Probability}
\subsection{Randomized algorithms}
Before we use randomized algorithm, we already know answer for the given input(we can find that out using bruteforce if we have a really fast computer for bruteforcing).\\
So to find out what the answer really is we take help of probabilities.\\
Since our classical computer cannot generate ideal randomness. Our algorithm cannot give a random answer(we can determine even before we run it).But it gives answer with bruteforce almost all the time (we can find test cases that doesn't though).

\section{Inclusion-Exclusion principle}
Consider any sets $A_1,A_2,\ldots A_n$ (discrete mathematical objects).
$$\abs{\bigcup\limits_{i=1}^{n} A_i} = \sum_{i=1}^n\abs{A_i} + \sum_{i<j}\abs{A_i\cap A_j} \ldots + (-1)^n\abs{A_1\cap \ldots \cap A_n}$$. As every element will contribute to both L.H.S and R.H.S by +1, Q.E.D.\\
Now in any problem, define the sets($A_i$'s) appropriately and apply above formula to get an useful result.\\

Important problems using inclusion-exclusion are:
\begin{itemize}
	\item finding number of subsets with certain gcd in an array.
	\item Using Mobius function is just same as using inclusion-exclusion internally.
	\item Number of solutions to $x_1+x_2+x_3 \ldots+x_r = n, 0\leq x_i \leq m \;\forall i$. 
\end{itemize}

\section{Dilworth's Theorem}
\begin{itemize}
	\item LIS can be calculated in $O(nlogn)$, using \href{./Material/LongestIncreasingSubsequence.pdf}{link}. It can be done using optimized DP anyway though. ($dp[i]=1+max(dp[j]) \: \forall j \ni a[j]<a[i]$). Where $dp[i]$ is LIS ending at $i$.
	\item There is another dp approach. $dp[i][j]=min(dp[i-1][j],a[i] \iff a[i]>dp[i-1][j-1])$. where $dp[i][j]$ is, using $[0\ldots i]$ elements, the minimum ending element among all the j length "increasing sequences".
	\begin{itemize}
		\item By induction on i we can prove that $dp[i]$ is sorted.
		\item $dp[i]$ differs from $dp[i-1]$ at atmost one position. That is the index $j_i$ where $j_i = max\{ j| dp[i-1][j-1] < a[i] \}$.
		\item We can update $dp[i-1]$ in-place to produce $dp[i]$.
	\end{itemize}
	\item To solve this problem in general. Create a bipartite graph with each element $x$ of poset on both sides ($u_x$ and $v_x$), add an edge $u_x \rightarrow v_y \iff x\leq y$ for($x\neq y$). Find the maximum matching m, then the minimum partition (width of poset) is $n-m$;
	\item Most of the problems can be solved by converting to LIS or LDS.
\end{itemize}



\end{document}