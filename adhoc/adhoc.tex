\documentclass[../Notes.tex]{subfiles}

\begin{document}
\chapter{Ad-hoc}
Any kind of observations, tricks etc are just used to decrease the time complexity and or memory complexity of the problem compared to brute force.(Similar to any standard  algo or ds you studied).\\

Ways to approach adhoc problem.
\begin{itemize}
	\item solve for smaller input by making one or multiple input dimensions 0 or 1 (may gives an idea or shows a pattern).
	\item GUESS
	\item If problem asks you to tell if any solution is not feasible. Identify the case when its not, probably rest of the cases will be possible.

	\item For the optimization problems or counting problems(count objects of some type), put multiple \textbf{EXAMPLES} on the paper and find the pattern.
	\item Similarly for the problems that require some quantity (nth element in array, sum of nth row in a matrix), when given relation between elements or rows to calculate them. But cannot do it due to time constraint. There may be a pattern present, write down \textbf{EXAMPLES} or write a bruteforce code for smaller n and output it to visualize the pattern.
	\item Optimization problems: Given a set of objects(graphs, integers, etc) implicitly. Find min or max among the attributes of the objects.\\ Example maximization problem, One solution is to prove that any object has attribute $<= p$. And construct the object with attribute p.  
\end{itemize}

Given two objects(both are of same class) A and B and an operation which transforms any object \textbf{X} to some object \textbf{Y}. Find minimum number of operations to convert A to B. Some ways to approach this problem.
\begin{itemize}
	\item Find an invariant(property of the object) that remains constant after a single operation.
	\item Find a monovariant(propert of the object) that changes by fixed amount after a single operation, no matter what the starting object is.
\end{itemize}

When printed all the constructions for inputs and they are huge in number, try to print only constructions satisfying some additional constraint. For example, construct a permutation with LIS = LDS  then simply print only constructions with LIS = 1 or LIS = n/2 etc.\\

Searching google is an art too. For example
\begin{itemize}
	\item For a pure mathematical problems like combinations and stuff, use their terminology. Use 'Sequences' instead of 'Arrays'.
\end{itemize}

\section{specific problems}
\begin{itemize}
	\item when asked to construct a new object where elements which are adjacent in initial object are not adjacent here, try to rotate(left or right shift) some part of the object(which may produce optimal answer). \\Ex. Produce a derangement.
\end{itemize}

\end{document}