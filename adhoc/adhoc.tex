\documentclass[../Notes.tex]{subfiles}

\begin{document}
\chapter{Ad-hoc}
Any kind of observations, tricks etc are just used to decrease the time complexity and or memory complexity of the problem compared to brute force.(Similar to any standard  algo or ds you studied).\\

Ways to approach adhoc problem.
\begin{itemize}
	\item For the optimization problems or counting problems(count objects of some type), put multiple \textbf{EXAMPLES} on the paper and find the pattern.
	\item Similarly for the problems that require some quantity (nth element in array, sum of nth row in a matrix), when given relation between elements or rows to calculate them. But cannot do it due to time constraint. There may be a pattern present, write down \textbf{EXAMPLES} or write a bruteforce code for smaller n and output it to visualize the pattern.
\end{itemize}
\end{document}