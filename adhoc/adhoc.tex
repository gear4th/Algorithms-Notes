\documentclass[../Notes.tex]{subfiles}

\begin{document}
\chapter{Ad-hoc}
Any kind of observations, tricks etc are just used to decrease the time complexity and or memory complexity of the problem compared to brute force.(Similar to any standard  algo or ds you studied).\\

Ways to approach adhoc problem.
\begin{itemize}
	\item For the optimization problems or counting problems(count objects of some type), put multiple \textbf{EXAMPLES} on the paper and find the pattern.
	\item Similarly for the problems that require some quantity (nth element in array, sum of nth row in a matrix), when given relation between elements or rows to calculate them. But cannot do it due to time constraint. There may be a pattern present, write down \textbf{EXAMPLES} or write a bruteforce code for smaller n and output it to visualize the pattern.
	\item Optimization problems: Given a set of objects(graphs, integers, etc) implicitly. Find min or max among the attributes of the objects.\\ Example maximization problem, One solution is to prove that any object has attribute $<= p$. And construct the object with attribute p.  
\end{itemize}

Given two objects(both are of same class) A and B and an operation which transforms any object \textbf{X} to some object \textbf{Y}. Find minimum number of operations to convert A to B. Some ways to approach this problem.
\begin{itemize}
	\item Find an invariant(property of the object) that remains constant after a single operation.
	\item Find a monovariant(propert of the object) that changes by fixed amount after a single operation, no matter what the starting object is.
\end{itemize}
\end{document}