\documentclass[../Notes.tex]{subfiles}


\begin{document}
\chapter{XOR}
Competitive problem setters love XOR (never understood the physical significance of it).\\

Different techniques to solve problems on XOR
\begin{itemize}
	\item Bit Trie.
	\item Solving for each individual bit and combining them at the end.
	\item Using Lexicographic property. If the most significant bit in binary form a number is 1 and the other number has 0 then first number is greater (no matter what the other bits are).
	\item In problems concerning XOR converting the array into prefix array can be useful.
\end{itemize}

Some tricks in solving XOR problems
\begin{itemize}
	\item $4k\oplus(4k+1)\oplus(4k+2)\oplus(4k+3)=0$ (used in finding XOR of numbers from 1..1e18)
	\item If $a\oplus b=0$ and $b\neq c$ $\implies$ $c\oplus b\neq0$. This trick is mostly used in number theory to prove theorems on nim.
	\item If the array is sorted, all the elements with the same first x bits will be contigous $\forall$x.
\end{itemize}

\pagebreak

\section{Game Theory}
There are two type of operations involved in game theory
\begin{itemize}
	\item MEX
	\item XOR
\end{itemize}

\textbf{Sprague-Grundy Theorem :} \\

If we know the MEX function of different games (G1, G2.. ) then the MEX function of combined games is XOR of all the functions.


\end{document}