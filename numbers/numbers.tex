\documentclass[../Notes.tex]{subfiles}

\begin{document}

\chapter{Numbers}

\begin{itemize}
	\item $\floor*{a/(bc)} = \floor*{\floor*{a/b}/c}$
	\item $\frac{1}{1}+\frac{1}{2}+\frac{1}{3}+...$ is divergent.
	Sum of first n elements \\
	$\ln(n+1)\le\sum_{i=1}^n\frac1i\le\ln(n)+1$
	\href{./Material/divergent series proof.pdf}{Proof link}
	\item the factors of any number are quite smaller compared to the  number. There doesn't exist a pair of distinct factors whose sum is $\geq$ the actual number.
	\item 'a' is multiple of $n_1$ and 'a' is multiple of $n_2$ iff 'a' is multiple of $LCM(n_1,n_2)$. 
	\item The set of all integer combinations of a and b is precisely the set of all integer multiples of the GCD of a and b. \href{https://proofwiki.org/wiki/Set_of_Integer_Combinations_equals_Set_of_Multiples_of_GCD}{proof}
	\item The above statement is generalized version of Bezout's identity.
\end{itemize}

\section{Modulo}
\begin{itemize}
	\item If you want to calculate $a^b$ modulo p(prime) where b is very large ($b=100000^{100000}$) we can use Fermat's \textbf{little} theorem. As $a^{p-1}\equiv 1(mod p)$, $a^b\equiv a^{bmod(p-1)}(mod p)$.  
	\item For prime p, and any number a relatively prime to p, a,2a,3a..p*a are all distinct modulo p.
	\item But $a^1,a^2,\dotsc a^{p-1}$ are not distinct modulo p. (a=6, p=5)
\end{itemize}

\section{CRT}
Find the solution set of $x\equiv a_1(mod n_1)$ and $x\equiv a_2(mod n_2)$.\\

Let $gcd(n_1,n_2)=d$ and find $x^\prime$ and $y^\prime$ s.t $n_1x^\prime + n_2y^\prime$ = d. (this is bezout's identity. Find it using Extended euclidean algorithm).\\
Note: d should divide $a_2-a_1$, otherwise there is no solution for the given two equations.

One of our solution is $x_o=a_1+x^\prime \frac{a_2-a_1}{d}n_1$. And our solution set is \\ $x_o+u*lcm(n_1,n_2), u\in Z$. This obtained solution set is equal to solution set of $x\equiv x_o(mod\;lcm(n1,n2))$.\\

For solving multiple equations combine first two equations in to one using above technique. Then again combine the newly obtained equation with the $3^{rd}$ one and so on.

\section{Euler's phi function}
\begin{itemize}
	\item Euler's phi function is multiplicative. \href{https://proofwiki.org/w/index.php?title=Euler_Phi_Function_is_Multiplicative&oldid=373969}{proof}.
	\item using above we can calculate $\varphi(n) = n\cdot \prod_{p\mid n}\left(1 - \frac1p\right).$ where p's are primes.
	\item Let any number k coprime to n then (k mod n) $\in \varphi(n)$.
	\item A more generalized form $\varphi (mn) = \varphi (m) \varphi (n) \cdot \dfrac{d}{\varphi (d)}$.
	\item if d\textbackslash n, the number of integers $k<=n$ and $GCD(k,n)=d$ are $\varphi(n/d)$.
\end{itemize}
\end{document}