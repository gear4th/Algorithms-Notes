\documentclass[../Notes.tex]{subfiles}


\begin{document}
\chapter{Arrays}
\begin{itemize}
	\item When we are asked to count number of pairs that satisfy a specific condition in an array. You should first try to rearrange the equation to get all the $i^{th}$ index terms to left side and $j^{th}$ index terms to right side. This will help us reduce $O(n^2)$ to $O(n)$.
\end{itemize}

\begin{itemize}
	\item There are 3 types(tentative) of sqrt decomp I know.
	\begin{itemize}
		\item MO's algorithm(offline algo) : When there is no query of update form then we can take all the queries and sort them in a certain way to make naive computation faster.
		\item (online algo)Divide the array in to blocks each of size ($\sqrt{n}$) and process both kind of queries in O($\sqrt{n}$) each.
		\item (online algo) Consider $\sqrt{q}$ operations at a time. Since there can be atmost $O(\sqrt{q})$ update operations. The array will be changed at atmost $\sqrt{q}$ indices hence if we find a way to answer the query operations (in this $\sqrt{q}$ block) with some preprocessing each in $O(\sqrt{q})$.\\
		 (Example. Find answer for this query without considering any updates in $O(1 \: or \: lgn)$ and update this answer according to changed indices in this range(which are only $\sqrt{q}$ in number). The time complexity of each block will be $O(q)$ hence total complexity will be $O(q\sqrt{q})$.(PreProcessing is done for each block seperately just before considering the block). \href{https://codeforces.com/problemset/problem/785/E}{Example Problem Link}.
	\end{itemize}
	\item Some problems involving subarrays, we can approach this problem through prefix sums,(donot forget that dp also works, even in max subarray sum). 
	\item Finding min element in subarray can be done using sparse table($O(nlgn)$ preprocess and $O(1)$ query). For 2d array ($O(nmlgnlgm)$ preprocess and $O(1)$ query).Where Seg tree takes $O(nm)$ preprocess and $O(lgnlgm)$ for query. (queries can be up to $n^2m^2$ in number).
	\item Queries of rectangles with fixed sizes can be done in $O(mn)$ pre and $O(1)$ query.
\end{itemize}

We can have pointers to the nearest equal, less or greater elements (either to left or right). This can help in certain problems
\begin{itemize}
	\item Find number of unique elements in range. More generally count number of elements in range, but if freq of an element is greater than 'k' take it only k.\\
	We can do it by having pointer to kth equal element to left. Then in range[l,r] find number of elements with $ptr[i]<l$, which is the answer.
\end{itemize}

Consider a divide and conquer algorithm with time complexity recurrence.
$T(n)=T(e_1)+T(e_2)+e_1 (e1<=e2 \wedge e1+e2=n)$.\\ 
The closed form is $T(n) = O(nlgn)$.\\
This can be proved using induction or realising the fact the every element is included in the $e_1$ term atmost lgn times.\\

\section{Segment Tree}
You are given an array of elements(may be null initially) and queries over subarrays.\\
There are two properties for a segment tree.
\begin{itemize}
	\item SegProp : required property from a segment.
	\item update type.
\end{itemize}

For the Segment tree to be valid it needs to satisfy
\begin{itemize}
	\item req 1 : We need to be able to find segprop from segprop's of any contigous partition subsegments of segment. 
\end{itemize}

For the lazy segment tree to be valid it needs to satisfy this below additional properties.
\begin{itemize}
	\item req 2 : If there are two or more updates on the same segment, we need to be able to combine them in to a single update.
	\item req 3 : If there is an update for a segment(whole segment). We need to be able to update segprop for the segment in $O(1)$.
\end{itemize}
\pagebreak
Lazy segment tree update template
\begin{lstlisting}
	void update(int si ,int ss, int se, int l, int r, int val){
		If there is lazy update pending:
			update seg prop for this node with lazy update;	\\req 3
			make lazy update for this node null;
			combine the lazy updates of children with this update;	\\req 2
		
		If current node segment is out of update range return;
		
		If the current node segment is completely in update range:
			update segprop of this node;	\\req 3
			combine the lazy updates of children with this update; \\req 2
			return;
		recursively call update for both children;
		update this node segprop with childrens seg prop. \\req 1
	}
\end{lstlisting}

Store the segprop's and lazy updates in 4*N sized array.\\

We can even store entire subsegment's in each node. The memory is only $O(nlgn)$. 

\section{Permutation}
\begin{itemize}
	\item If we have a directed graph of n nodes, and connect ith to jth iff $p[i]=j$; then we get a bunch of disjoint cycles.
	\item Let p be permutation and $next_{i}$ be the smallest $j>i$ s.t $p[j]>p[i]$. Then when we draw edges(curves) from $i -> next_{i}$ on top of the array then no two edges intersect. (i.e the condition $i<j<next_{i}<next_{j}$ is never satisfied for any pair of indices i,j).
	\item You can find number of inversions in an array using policy data structure (balance BST). Infact we can do many variations of inversion problem. (Example. Find number of inversions when equal elements are present, Which is not possible using merge sort.) 
\end{itemize}

\end{document}